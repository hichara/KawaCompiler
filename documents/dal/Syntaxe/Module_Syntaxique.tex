\documentclass{../res/univ-projet}

%Import des packages utilisés pour le document
\usepackage[utf8x]{inputenc}
\usepackage[francais]{babel}
\usepackage[T1]{fontenc}
\usepackage{listings}
\usepackage[T1]{fontenc}
\makeatletter
\newcommand\verbfile[1]{%
	\begingroup
		\let\do\@makeother\dospecials
		\parindent\z@\obeyspaces\ttfamily
		\catcode`\^^M\active
		\begingroup\lccode`\~`\^^M \lowercase{\endgroup\def~{\par\leavevmode}}%
		\input#1\relax
	\endgroup
}
\makeatother

%\usepackage{array}
%\usepackage{hyperref}
%\usepackage{tabularx, longtable}
%\usepackage[table]{xcolor}
%\usepackage{fancyhdr}
%\usepackage{lastpage}

\definecolor{gris}{rgb}{0.95, 0.95, 0.95}
\definecolor{listinggray}{gray}{0.4}
\definecolor{lbcolor}{rgb}{0.91,0.91,0.91}
\lstset{
	backgroundcolor=\color{lbcolor},
	tabsize=3,
	rulecolor=,
	%fillcolor=\color{lbcolor},,
        basicstyle=\scriptsize,
        %basicstyle=\ttfamily,
        upquote=true,
        aboveskip={0.8\baselineskip},
        %columns=fixed,
        showstringspaces=false,
        numbers=left ,numberstyle=\tiny\bfseries ,
        stepnumber=1,
        firstnumber=1,
        numberfirstline=true
    	numbersep=16pt,
        extendedchars=true,
        breaklines=true,
        columns=flexible,
        prebreak = \raisebox{0ex}[0ex][0ex]{\ensuremath{\hookleftarrow}},
        frame=single,
        %frameround=ffff,
        %framerule=0.05cm,
        showtabs=false,
        showspaces=false,
        showstringspaces=false,
        identifierstyle=\ttfamily,
        %keywordstyle=\color[rgb]{0,0,1},
        keywordstyle=\bfseries\ttfamily\color[rgb]{0,0,1},
        commentstyle=\color[rgb]{0.133,0.545,0.133},
        %stringstyle=\color[rgb]{0.627,0.126,0.941},
        stringstyle=\ttfamily\color[rgb]{0.627,0.126,0.941},
}

%Redéfinition des marges
%\addtolength{\hoffset}{-2cm}
%\addtolength{\textwidth}{4cm}
\addtolength{\topmargin}{-1cm}
\addtolength{\textheight}{1cm}
\addtolength{\headsep}{0.8cm} 
\addtolength{\footskip}{-0.2cm}


%Import page de garde et structures pour la gestion de projet
%\usepackage{structures}

%Variables
\logo{../res/logo_univ.png}
\title{Plan de développement}
\author{Pierre-Luc BLOT, Kheireddine \bsc{Berkane}}
\projet{Compilateur LLVM}
\projdesc{Langage jouet Kawa}
\filiere{M1GIL - Conduite de Projet}
\version{0.1}
\relecteur{Nasser \bsc{ADJIBI}}
%\signataire{Florent \bsc{NICART}}
\date{\today}

\histentry{0.1}{23/12/2014}{Version initiale.}


% -- Début du document -- %
\begin{document}

%Page de garde
\maketitle
\newpage
%La table des matières
\tableofcontents
\newpage
\section{Analyse lexicale}
\subsection{Objectif de l'analyse lexicale}
le principale objectif de cette analyse est de reconnaitre des unités lexicales (tokens).
Chaque token est défini par une expression rationnelle, les entités reconnues lors de cette phase serviront par la suite comme entrée de la phase d'analyse syntaxique.

\subsection{Lexer}
\verbfile {lexer.txt}



\newpage
\section{Analyse syntaxique}
\subsection{Objectif de l'analyse syntaxique}
% Présentation succinte du sujet et hyp de travail.
l'objectif de l'analyse syntaxique est de reconnaître les phrases appartenant à la syntaxe du langage
son entrée est le flot des lexèmes construits par l'analyse lexicale, sa sortie est un arbre de syntaxe abstraite représentant le programme. Elle permet d'identifier et de localiser les erreurs de syntaxe du langage pour notre cas le langage source est le langage jouet kawa.\\

\subsection{Grammaire du langage kawa}
Afin de spécifier et de générer le langage kawa nous avons défini une grammaire non ambigüe permettant d'engendrer toute la syntaxe du langage source, cette grammaire permettra par la suite de construire l'arbre de syntaxe abstraite qui lui sera utile pour la prochaine phase d'analyse sémantique.\\
Notre grammaire est constituée d'un ensemble de règles de productions :\\

% \input{grammaire.txt} %inclus par \jobname .tex
\verbfile {grammaire.txt}
% Bloc Grammaire


%END Bloc grammaire

\subsection{Construction de l'arbre abstrait} 
 Nous construisons à travers cette phase d'analyse un arbre abstrait de syntaxe qui sera utilisé par les prochaines phases d'analyse (la sémantique et la génération de code intermédiaire). L'analyse sémantique a pour objectif de vérifier le sens de l'arbre monté en mémoire et de le décorer en ajoutant des informations nécessaires pour la génération de code intermédiaire.\\
 
Dans le cadre de ce projet nous avons opté pour une solution qui produit un seul arbre en mémoire \textbf{KawaTree} qui est une collection de classe. Ce même KawaTree sera utilisé par l'ensemble des modules, chacun de ces modules possèdent une interface de connexion avec l'arbre pour extraire les informations nécessaires à cette phase d'analyse.\\

Notre objectif pendant cette phase est de construire et de produire le KawaTree afin de permettre aux autres modules de continuer la chaine de compilation en prenant en paramètre d'entrée le KawaTree qui est un arbre abstrait commun pour tous les modules

Nous présentons le diagramme de classe (KawaTree)ainsi que des classes qui ont une relation avec ce diagramme mais qui ne sont utilisées que par le module syntaxique:

\begin{figure}[h!]
\centering
\includegraphics[scale=0.20]{parser_ktdiagramme.png}
\caption[KawaTree avec des classes du parser.]{KawaTree avec des classes du parser.}
\end{figure}
  
  \newpage
 \subsection{Outils de réalisation} 
 Afin de réaliser le module d'analyse syntaxique nous avons utilisé les deux outils:
 \begin{itemize}
 \item  \textbf{Flex}: c'est une version de lex qui est un générateur d'analyse lexical, nous pouvons définir à travers cet outil des unités lexicales pour les reconnaitre après dans un processus de compilation d'un programme source.
\item \textbf{GNU Bison}: est l'implémentation GNU du compilateur de compilateur yacc, spécialisé dans la génération d'analyseurs syntaxiques, il permet de définir la grammaire du langage source ainsi que de produire un analyseur syntaxique en déclenchant les actions des règles de productions invoquées par le programme source, les règles de production sont déclenchés de bas vers le haut car bison est basé sur un analyse ascendante c'est la méthode d'analyse la plus performante.
 
\end{itemize}   ~\\
Nous utiliserons ces deux outils Flex et Bison dans cette phase d'analyse syntaxique car le bison prend en paramètre d'entré les lexèmes qui ont été défini dans Flex afin de produire un arbre abstrait de syntaxe ce dernier sera généré en c++.


\end{document}

