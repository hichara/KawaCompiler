\documentclass{../res/univ-projet}

%Import des packages utilisés pour le document

% minted : http://www.ctan.org/tex-archive/macros/latex/contrib/minted/
% pour installer sur ubuntu:
% http://tex.stackexchange.com/questions/40083/how-to-install-minted-in-ubuntu#40101

\usepackage{lscape}
\usepackage{listings}
\usepackage{minted}
\usepackage{multicol}
\usepackage{color}



\definecolor{success}{HTML}{198C19}
\definecolor{danger}{HTML}{CC0000}
\definecolor{bg-gray}{HTML}{F9F9F9}

\definecolor{gris}{rgb}{0.95, 0.95, 0.95}

%Redéfinition des marges
\addtolength{\topmargin}{-1cm}
\addtolength{\textheight}{1cm}
\addtolength{\headsep}{0.8cm} 
\addtolength{\footskip}{-0.2cm}




%Variables
\logo{../res/logo_univ.png}
\title{Manuel d'utilisation}
\author{Pierre-Luc BLOT}
\projet{Compilateur LLVM}
\projdesc{Langage jouet Kawa}
\filiere{M1GIL}
\version{0.1}
\relecteur{}
%\signataire{Florent \bsc{NICART}}
\date{\today}

\histentry{0.1}{19/04/2015}{Version initiale.}


% -- Début du document -- %
\begin{document}

%Page de garde
\maketitle
\newpage
%La table des matières
\tableofcontents
\newpage

\section{Description du compilateur kawac} 
L'outil \textbf{kawac} lit des définitions de classes et des interfaces écrites dans le
langage de programmation Kawa et les compiles en executable ELF. Il est aussi
possible de les compiler en bytecode.

  \subsection{compilation}
    Il y a deux manières de passer des noms de fichiers sources à \textbf{kawac}.
    \begin{itemize}
      \item L’utilisateur introduit un ensemble de modules (classe/interface)
      kawa afin de les pré/compiler et de générer une application sous forme
      d’un seul exécutable.\\

      Afin de permettre la compilation dans ce mode monolithique il faut :
      \begin{itemize}
        \item Indiquer l’emplacement des fichiers sources de l’application.
        \item Introduire le commutateur -m en premier paramètre dans la
        ligne de commande permettant la compilation.
        \item L’absence des deux commutateurs (\textbf{-h} ou \textbf{--help}) et (\textbf{-v} ou \textbf{-
        -version}) en premier paramètre.
        
      \end{itemize}

      Afin de pouvoir lancer la compilation dans ce mode l’utilisateur doit :
      \begin{itemize}
        \item Indiquez au moins un module (une classe avec une méthode
        main qui constitue le point d’entrée d’une application).
        \item Tapez la ligne de commande suivante afin de compiler les sources kawa :\\
        \begin{minted}[bgcolor=bg-gray]{bash}
          :$ kawac -m filessources
        \end{minted}
        
        
      \end{itemize}

      Cela déclanche la production d’un unique fichier exécutable sous le format
      d’ELF qui ne dépend d’aucune bibliothèque kawa ainsi que le nom du fichier 
      exécutable correspond au nom du module contenant la méthode main.\\

      La compilation peut être interrompue pour divers raisons, on cite :
      \begin{itemize}
        \item Le fichier de sortie n’a pu être créé.
        \item Le code source d’un des modules de l’application
        comporte une erreur syntaxique, l’erreur rencontrée pendant cette
        étape d’analyse est renvoyée sur la sortie standard du compilateur.
        \item Le code source d’un des modules
        (classe/interface) dont dépend le programme est introuvable.
        Un message renvoyé par le compilateur indique le nom du (ou des)
        module(s) manquants sur la sortie standard.
        \item Le code source d’un des modules de l’application
        comporte une erreur sémantique, exemple : l’incompatibilité de type
        statique lors d’une opération d’affectation.
        \item Aucun des modules indiqués ne comporte le
        point d’entrée (main).
        \item Plusieurs méthodes main ont été trouvées parmi
        les classes indiquées dans le source, le compilateur renvoie la liste
        des points d’entrée trouvés sur la sortie standard.
        \item Notez qu'il est important que vos définitions de \textbf{package} soient bien situées
         dans des fichiers se trouvant dans des répertoires ayant le nom de ce \textbf{paquetage}.
      \end{itemize}

    \end{itemize}
  \subsection{afficher la version du compilateur}
    Il est possible d'afficher l'aide du compilateur avec le commutateur \textbf{-h} ou \textbf{--help}. Cela permet d'obtenir rapidement une aide quant à la manière d'utiliser \textbf{kawac} et ce, depuis le terminal via une ligne de commande.
  \begin{minted}[bgcolor=bg-gray]{bash}
    :$ kawac --help
  \end{minted}

\newpage
  \subsection{afficher la version du compilateur}
    Il est possible de savoir la version du compilateur avec le commutateur \textbf{-v} ou \textbf{--version}
    \begin{itemize}
      \item L’odre de priorité entre le commutateur de help (\textbf{-h} ou \textbf{--help}) et le commutateur de version (\textbf{-v} ou \textbf{--version}) est définit par la première occurrence de l’un des deux commutateur i-e si nous avons un \textbf{-v}
      avant un \textbf{-h}, le compilateur annule tout le reste et affiche la version
  \end{itemize}
  \begin{minted}[bgcolor=bg-gray]{bash}
    :$ kawac --version
  \end{minted}

 \section{Dépendances de kawac}
  Le compilateur kawa \textbf{kawac} utilise la technologie \textbf{LLVM} afin de produire du bytecode. \textbf{kawac} offre un frontend \textbf{LLVM} pour le langage Kawa ainsi qu'une partie du backend ce qui produit du bytecode. Néanmoins, pour produire un exécutable à partir du bytecode \textbf{kawac} a besoin d'\textbf{llc} dans sa version 3.4/3.5 et de \textbf{gcc} dans sa version 4.8.2 ou plus.

	\subsection{Installation des dépendances}
		Afin de compiler les sources de kawac, il est nécessaire d'installer les paquets suivants:
		\begin{itemize}
			\item bison
			\item flex
			\item llvm-3.5
			\item llvm-3.5-tools
			\item clang-3.5
			\item zlib1g-dev 
			\item libedit-dev
		\end{itemize}
                \begin{minted}[bgcolor=bg-gray]{bash}
:$ sudo apt-get install bison flex llvm-3.5 llvm-3.5-tools clang-3.5 zlib1g-dev libedit-dev
\end{minted}

		Il est possible que la commande llc ne soit pas installée automatiquement. Si tel est le cas il sera nécessaire de le faire manuellement:\\
                \begin{minted}[bgcolor=bg-gray]{bash}
:$ sudo ln -s /usr/bin/llc-3.5 /usr/bin/llc
\end{minted}


 \section{Fonctionnement de la chaine de compilation}
  Le frontend de kawac se nomme kawap (pour kawa bytecode production) qui utilise la technologie \textbf{LLVM} afin de produit du bytecode. Le compilateur en lui même se nomme \textbf{kawac} et encapsule le frontend ainsi que le backend. Voici comment se déroule la compilation d'un exécutable avec la commande:\\
  \begin{minted}[bgcolor=bg-gray]{bash}
    :$ kawac -m filessources
  \end{minted}

  \begin{enumerate}
    \item kawap -m filessources $\rightarrow$ produit du bytecode dans un fichier .ll
    \item llc $\rightarrow$ qui produit du code assembleur à partir du bytecode
    \item gcc $\rightarrow$ qui crée le fichier exécutable à partir de l'assembleur
  \end{enumerate}

  L'exécutable \textbf{kawac} est un script bash qui explicite cette chaîne de compilation et vérifie les dépedances automatiquement.
\end{document}
