\section{Cahier de recettes}
     
   %%%%
   \subsection{Mise en place}
      \begin{frame}{Cahier de recettes}{Mise en place}
        \begin{block}{Mise en place}
          Lancement d'une phase de test après chaque fonctionnalité implantée:
          \begin{itemize}
            \item<1-> Mise en place de tests spécifiques
            \item<2-> Vérification du bon fonctionnement
            \item<3-> Possibilité de mise en place de tests de non-régression
            \item<4-> Retour développeur
            \item<5-> Mise à jour cahier de recettes
          \end{itemize}
        \end{block}
      \end{frame}
      %%%%
   \subsection{Spécificité des tests}
      \begin{frame}{Cahier de recettes}{Spécificité des tests}
        \begin{block}{Test Unitaire}
          \begin{itemize}
            \item<1-> Objectif: Vérifier la validité d'un module de code.
            \begin{itemize}
              \item<2-> Fonctionnalité
              \item<3-> Exigence précise
            \end{itemize}
          \end{itemize}
        \end{block}
      \end{frame}
      %%%%
      \begin{frame}{Cahier de recettes}{Spécificité des tests}
        \begin{block}{Test d'intégration}
          \begin{itemize}
            \item<1-> Objectif: Vérifier la validité d'un ensemble de fonctionnalités afin d'en faire un livrable.
          \end{itemize}
        \end{block}
      \end{frame}
      %%%%
      \begin{frame}{Cahier de recettes}{Spécificité des tests}
        \begin{block}{Test de non-régression}
          \begin{itemize}
            \item<1-> Objectif: Vérifier que l'ajout de fonctionnalités n'a pas eu de répercussion sur l'existant.
            \item<2-> Pas toujours nécessaire.
          \end{itemize}
        \end{block}
      \end{frame}

    \subsection{Approche d'écriture des tests}
      \begin{frame}{Cahier de recettes}{Approche d'écriture des tests}
        \begin{block}{Réalisation}
          \begin{itemize}
            \item<1-> Écriture automatisée
            \item<2-> Écriture du code pour lancer tests
            \item<3-> Vérification des résultats
          \end{itemize}
        \end{block}
        \begin{block}{Objectifs}
          \begin{itemize}
            \item<4-> Augmenter la complexité des tests facilement
            \item<5-> Éviter la redondance dans l'écriture des tests
            \item<6-> Valider avec une plus grande certitude le bon déroulement des fonctions.
          \end{itemize}
        \end{block}
      \end{frame}

      \subsection{Objectifs des tests}
      \begin{frame}{Cahier de recettes}{Objectifs des tests}
        \begin{block}{Objectifs des tests}
          \begin{itemize}
            \item<1-> Vérifier que chaque fonctionnalité réalise sa tâche correctement.
            \item<2-> Valider des modules.
            \item<3-> Réaliser une large couverture de code.
          \end{itemize}
        \end{block}
      \end{frame}  
  %%%%%%%%%%%%%%%