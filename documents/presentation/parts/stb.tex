\section{Analyse des besoins du client}
    \begin{frame}{Analyse des besoins du client}
      \begin{itemize}
        \item<1-> Langage imposé {\tt Kawa}
        \item<2-> Technologie imposée {\tt LLVM}
        \item<3-> Pourquoi kawa ?
          \begin{itemize}
            \item<4-> Langage qui est similaire à java.
            \item<5-> Bénéficier des fonctionnalités d'un langage de haut niveau ainsi que des performances bas niveau.
            \item<6-> Compiler kawa en code natif sans passer par une machine virtuelle.
          \end{itemize}
        
      \end{itemize}
    \end{frame}


  \subsection{Notions supportées par les modes de compilation}
    \begin{frame}{Analyse des besoins du client}{Notions supportées par les modes de compilation}
      \begin{block}{Notions supportées par les modes de compilation}
        \begin{itemize}
          \item<1-> Classe, classe abstraite et interface.
          \item<2-> Héritage
          \item<3-> Polymorphisme
          \item<4-> Contrôle de types
        \end{itemize}
      \end{block}
    \end{frame}

    \begin{frame}{Analyse des besoins du client}{Notions supportées par les modes de compilation}
      \begin{block}{Notions supportées par les modes de compilation}
        \begin{itemize}
          \item<1-> Compilation d'une application en mode monolithique
          \item<2-> Compilation d'une application en mode partagé.
          \item<3-> Compilation d'une bibliothèque partagée.
        \end{itemize}
      \end{block}
    \end{frame}
%%%%%%%%%%%%%%%%