\section{Test}
    \begin{frame}{Test}    	
		\begin{itemize}
			\item Pourquoi des Tests?
		\end{itemize}
	
    \end{frame}

    \subsection{Stratégie des Tests}
	\begin{frame}{Stratégie des Tests}{Pour valider les procédures de tests}
		\begin{block}{Pour valider les procédures de tests}
			\begin{enumerate}
				\item<1-> Tests Unitaires.
				\item<2-> Tests d'intégrations.
				\item<3-> Tests de non regressions.
			\end{enumerate} 
			\pause
			\pause
			\pause
			
		\end{block}
		\begin{block}{NB :} % Bloc normal
			A chaque fois qu’un test est réalisé il fallait notifier\\
		    le résultat obtenu dans le tableau des procédures\\
           (se trouvant dans le cahier de recette).
            L’objectif étant d’avoir tous les tests validé à la fin du projet.\\
         	Et que ces tests touchent le plus de ligne de code possible; \\
        	 c’est ce qu’on appelle la couverture de code. 
		\end{block}
	\end{frame}

	\subsection{Bilan Tests}
		\begin{frame}{Bilan Tests}{Différents Modules}
			\begin{enumerate}
				\item<1-> Module Syntaxique.
				\item<2-> Module Sémantique.
				\item<3-> Structure KawaTree.
			\end{enumerate} 		
		\end{frame}

	%Gestion des Anomalies
	\subsection{Gestion des Anomalies}
		
		\begin{frame}{Gestion des Anomalies}{Echec Test}
			\begin{block}{Que doit on faire?}
				\begin{itemize}
					\item<1-> Echec d'un test.
					\item<2-> Résolution du problème.
					\item<3-> Validation de la résolution du problème
				\end{itemize} 
			\end{block}
		\end{frame}

	%Automatisation des Tests
	\subsection{Automatisation des tests}
		\begin{frame}{Automatisation}{Difficultés rencontrées}
			\begin{itemize}
				\item<1-> Prise en main de CppUnit.
			\end{itemize} 
		\end{frame}

