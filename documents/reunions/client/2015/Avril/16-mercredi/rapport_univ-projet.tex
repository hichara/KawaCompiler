\documentclass{../../../../../res/rapport}

\usepackage[T1]{fontenc}
%\usepackage[utf8]{inputenc}
%\usepackage{DejaVuSans}

\logo{../../../../../res/logo_univ.png}
\author{AHOUATE Abdellatif}
\participants{Ahouate Abdellatif,
			  Alexandre Petre,
              Idrissou Amzath,  
			  Kheireddine Berkane}
\absents{Nasser Adjibi, Pierre-luc Blot}
\filiere{M1GIL}
\projet{Compilateur LLVM}
\projdesc{Langage jouet KAWA} % si la description est trop longue pensez 
                              % à mettre \\ pour un retour à la ligne
\title{Compte rendu du Mercredi 15 Avril 2015}
\version{n°1}

%\histentry{0.5}{10/12/2012}{Changelog pour la d\'emo}

\begin{document}
    \maketitle
    \tableofcontents
    \clearpage
    
    \section{Observations et questions} 
    \label{sec:observations_et_questions}
	\begin{itemize}
        \item On a parlé de l'organisation de l'équipe.
	\item Le client a exigé un push chaque jour pour que le chef de projet Kheireddine Berkane pourra suivre l'avancement en permanent du projet pour ce qui reste du projet.
	\item Le client a exigé un mini livrable pour le Dimanche 19 Avril 2015, ce dernier reprend la notion du polymorphisme fonctionnel; programme qui contient:
        	\begin{itemize}
        		\item Une interface avec une méthode, et deux classes implémentant l'interface ainsi qu'une classe Test. 
			\item Chacun des deux classes contient qu'une méthode sans body.
			\item Chacun des deux classes contient qu'une méthode sans body.
			\item la classe Test contient la méthode main pour tester l'appelle polymorphe.
        	\end{itemize} 
	\item On a détaillé ce qu'il faut faire pour chaque module.
		\begin{itemize}
        		\item Partie Front end:
 				\begin{itemize}
        				\item L'intégration de la partie nécessaire de KawaTree pour le mini livrable dans le parser. 
					\item Gérer plusieurs fichiers en entrées. 
					
        			\end{itemize}
			\item Partie analyse sémantique:
				\begin{itemize}
        				\item Enregistrer chaque classe comme un type.
					\item Définir les relations d"héritage (une seule interface).
					\item Décorer les classes et les méthodes.
					\item Traiter Main et Body. 					
        			\end{itemize}
			\item Partie Back-end : On a rien détaillé puisque Nasser Adjibi était absent 
			
        	\end{itemize} 
		
	\end{itemize}
\end{document}
