\documentclass{../../../../../res/rapport}

\usepackage[T1]{fontenc}
%\usepackage[utf8]{inputenc}
%\usepackage{DejaVuSans}

\logo{../../../../../res/logo_univ.png}
\author{AHOUATE Abdellatif}
\participants{Kheireddine Berkane,
              Pierre-luc Blot,
              Nasser Adjibi , 
              Ahouate Abdellatif,
              Idrissou Amzath}
\absents{Alexandre Petre,
         Madjid Tasserie}
\filiere{M1GIL}
\projet{Compilateur LLVM}
\projdesc{Langage jouet KAWA} % si la description est trop longue pensez 
                              % à mettre \\ pour un retour à la ligne
\title{Compte rendu du Mercredi 19 novembre 2014}
\version{n°4}

%\histentry{0.5}{10/12/2012}{Changelog pour la d\'emo}

\begin{document}
    \maketitle
    \tableofcontents
    \clearpage
    
    \section{Observations et questions} 
    \label{sec:observations_et_questions}
        \begin{itemize}
            \item La première version de la STB 0.1 est finalisée.\\

            \item Commencer la rédaction de la première version du CDR 0.1.\\

            \item Création d'un aide mémoire GIT\\
            \item Création d'un fichier qui décrit la routine de travail Git\\
            \item Ordres de la réunions client
            	\begin{itemize}
            		\item utilisation de super
            		\item utilisation de la méthode main. Avec paramètres ?
            	\end{itemize}
            
        \end{itemize}
        
    \section{Plan d'actions} 
    \label{sec:plan_d_actions}
        \begin{itemize}
            \item Prochaine réunion avec le client est pour le Jeudi 20 Nomembre
                  à 10h15-12h15 
            \item Prochaine réunion en groupe est pour le Vendredi 19 Nomvembre
                   à 10h15
            \item Nous devons préciser comment installer LLVM 3.5 sur tout le 
                  système et non dans le répértoire isolé.
            \item Les membres du groupe devront être prêt avec Latex et Git.
        \end{itemize}
        
        
\end{document}
