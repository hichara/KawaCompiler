\documentclass{../../../../../res/rapport}

\usepackage[T1]{fontenc}
%\usepackage[utf8]{inputenc}
%\usepackage{DejaVuSans}

\logo{../../../../../res/logo_univ.png}
\author{AHOUATE Abdellatif}
\participants{Kheireddine Berkane,
              Pierre-luc Blot,
              Ahouate Abdellatif
              }
\absents{Alexandre Petre,
         Madjid Tasserie,
         Nasser Adjibi,
         Idrissou Amzath}
\filiere{M1GIL}
\projet{Compilateur LLVM}
\projdesc{Langage jouet KAWA} % si la description est trop longue pensez 
                              % à mettre \\ pour un retour à la ligne
\title{Compte rendu du Jeudi 20 novembre 2014}
\version{n°5}

%\histentry{0.5}{10/12/2012}{Changelog pour la d\'emo}

\begin{document}
    \maketitle
    \tableofcontents
    \clearpage
    
    \section{Observations et questions} 
    \label{sec:observations_et_questions}
        \begin{itemize}
            \item Il faut apporter les modifications sur le diaramme de cas d'utilisations :
                  \begin{itemize}
                      \item Séparer le cas d'utilisation "compile application monolithique" avec "compiler en partagé"
                      qui sera nommé "compiler appilcation partagé".

                      \item Ajouter le cas d'utilisation "compiler application en utilisant des bibliothèques partagées" qui sera lié avec l'utilisateur.
                      \item Les cas d'utilisations "Définir dépendances" qui sera nommé "Indiquer le chemain de dépendance" et "Activer affichage en couleur" ne seront plus liés avec l'utilisateur.
                      \item Supprimer le cas d'utilisation "choisir nom executable".\\
                  \end{itemize}

            \item Il faut retirer le cas d'utilisation EF-4.\\

            \item Il faut préciser la ligne de commande dans "Evénements déclenchants" pour chaque cas d'utilisation. \\
            \item En mode de compilation partagé : 
                  \begin{itemize}
                      \item les sous packages sont autorisés (en priorie).
                      \item La compilation sera fait par classe (La création de .so pour chaque classe) pas par package (La création de .so pour chaque package) par suite de plusieurs avantages :
                            \begin{itemize}
                                  \item L'arborescence sera plus élégant.
                                  \item L'arborescence binaire sera identique à clle du code.
                                  \item La compilation sera unitaire.
                                  \item La détection de la modification de code sera plus simple.
                            \end{itemize}
                      \item Les testes des performance pour savoir quel mode de compilation choisir sont secondaires (Compilation par classe ou par package). \\
                  \end{itemize}

            \item L'affichage des messages en couleur sera ultra secondaire.\\
            \item Ligne de commande pour la comilation sera comme suite : kawac [file-source][options] \\
                  Liste de options autorisés:
                  \begin{itemize}
                      \item -d : Suivie par le chemain de dépendance (obligatoir pour chaque dépendance). 
                      \item -h ou --help : Affiche l'aide.
                      \item -v ou --vesrion : Affiche la version.
                      \item -m : Monolithique.\\
                  \end{itemize}
            
        \end{itemize}
        
    \section{Plan d'actions} 
    \label{sec:plan_d_actions}
        \begin{itemize}
            \item Prochaine réunion en groupe est pour le Vendredi 21 Nomvembre
                   à 10h15
            \item Prochaine réunion avec le client est pour le Jeudi 27 Nomembre
                  à 10h15-12h15 
            \item Nous devons préciser comment installer LLVM 3.5 sur tout le 
                  système et non dans le répértoire isolé.
            \item Les membres du groupe devront être prêt avec Latex et Git.
        \end{itemize}
        
        
\end{document}
