\documentclass{../../../../../res/rapport}

\usepackage[T1]{fontenc}
%\usepackage[utf8]{inputenc}
%\usepackage{DejaVuSans}

\logo{../../../../../res/logo_univ.png}
\author{AHOUATE Abdellatif}
\participants{Kheireddine Berkane,
              Pierre-luc Blot,
              Madjid Tasserie, 
              Ahouate Abdellatif,
              Idrissou Amzath}
\absents{Alexandre Petre,
         Nasser Adjibi}
\filiere{M1GIL}
\projet{Compilateur LLVM}
\projdesc{Langage jouet KAWA} % si la description est trop longue pensez 
                              % à mettre \\ pour un retour à la ligne
\title{Compte rendu du Vendredi 21 novembre 2014}
\version{n°6}

%\histentry{0.5}{10/12/2012}{Changelog pour la d\'emo}

\begin{document}
    \maketitle
    \tableofcontents
    \clearpage
    
    \section{Observations et questions} 
    \label{sec:observations_et_questions}
        \begin{itemize}
            
            \item Amélioration de la classe \LaTeX pour la STB
            \item Préparation/Formation GIT
            \item Préparation/Formation \LaTeX
            	 
        \end{itemize}
        
    \section{Plan d'actions} 
    \label{sec:plan_d_actions}
        \begin{itemize}
            \item Prochaine réunion en groupe est pour le Mardi 26 Nomvembre à 13h30-14:30
            \item Prochaine réunion avec le client est pour le Jeudi 20 Nomembre
                  à 10h15-12h15
            \item Pour les résponsables téchniques :
                \begin{itemize}
                  \item Vérifier que les exemples LLVM 3.5 fonctionnent.
                  \item Trouver un moyenne d'installer les sources compilées sur tout le système et non dans un répertoire isolé (Au plus tard le Mercredi 26/11/2014).
                  \item Ercire une notice de téléchargement des sources et de compilation.
                  \item La bon fonctionnement sur ubuntu 14.04 (64 bits en priorité et 32 bits) et 14.10 (64 bits en priorité et 32 bits) doit être garanti.
                \end{itemize}
        \end{itemize}
        
        
\end{document}
