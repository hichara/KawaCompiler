\documentclass{../../../../../res/rapport}

\usepackage[T1]{fontenc}
%\usepackage[utf8]{inputenc}
%\usepackage{DejaVuSans}

\logo{../../../../../res/logo_univ.png}
\author{AHOUATE Abdellatif}
\participants{Kheireddine Berkane,
              Pierre-luc Blot,
              Majid Tasserie, 
              Ahouate Abdellatif,
              Idrissou Amzath,
              Alexandre Petre,
         	  Nasser Adjibi}
\absents{}
\filiere{M1GIL}
\projet{Compilateur LLVM}
\projdesc{Langage jouet KAWA} % si la description est trop longue pensez 
                              % à mettre \\ pour un retour à la ligne
\title{Compte rendu du Mardi 25 novembre 2014}
\version{n°8}

%\histentry{0.5}{10/12/2012}{Changelog pour la d\'emo}

\begin{document}
    \maketitle
    \tableofcontents
    \clearpage
    
    \section{Observations et questions} 
    \label{sec:observations_et_questions}
        \begin{itemize}
            
            \item La fonctionnement des exemples de LLVM-3.5 sur ubuntu 14.04 (32 et 64 bits) et ubuntu 14.10(64 bits).
            \item La notice d'installation de LLVM 3-5 est faite.
            \item le commencement de la construction de la gramaire
            	 
        \end{itemize}
        
    \section{Plan d'actions} 
    \label{sec:plan_d_actions}
        \begin{itemize}
            \item Prochaine réunion en groupe est pour le vendredi 28 Nevembre à 10h15-12h15.
            \item Prochaine réunion avec le client est pour le Jeudi 4 Décembre.
                  à 10h15-12h15
            \item Pour les résponsables téchniques :
                \begin{itemize}
                  \item Construire la grammaire qui engendre le langage Kawa.
                  \item Construire AST (l'arbre sytaxique). 
                  \item Décrire c'est quoi le front-end et ses étapes.
                  \item Décrire c'est quoi le back-end et ses étapes.
                \end{itemize}
        \end{itemize}
        
\end{document}
