\documentclass{../../../../../res/rapport}

\usepackage[T1]{fontenc}
%\usepackage[utf8]{inputenc}
%\usepackage{DejaVuSans}

\logo{../../../../../res/logo_univ.png}
\author{AHOUATE Abdellatif}
\participants{Kheireddine Berkane,
              Idrissou Amzath,  
              Ahouate Abdellatif,
         	  Nasser Adjibi}
\absents{Pierre-luc Blot,
			Majid Tasserie, 
            Alexandre Petre}
\filiere{M1GIL}
\projet{Compilateur LLVM}
\projdesc{Langage jouet KAWA} % si la description est trop longue pensez 
                              % à mettre \\ pour un retour à la ligne
\title{Compte rendu du Mardi 10 Fevrier 2015}
\version{n°2}

%\histentry{0.5}{10/12/2012}{Changelog pour la d\'emo}

\begin{document}
    \maketitle
    \tableofcontents
    \clearpage
    
    \section{Observations et questions} 
    \label{sec:observations_et_questions}
	Séance de travail sur LLVM, les points abordés sont :
            \item Créer le prototype d'une méthode (valeur de retour et les paramètres)
			\item Créer le corp d'une méthode
			\item Faire une simple affectation avec le caste genre int x=5+3.2
			\item Créer des structures 		
\end{document}
