\documentclass{../../../../../res/rapport}

\usepackage[T1]{fontenc}
%\usepackage[utf8]{inputenc}
%\usepackage{DejaVuSans}

\logo{../../../../../res/logo_univ.png}
\author{AHOUATE Abdellatif}
\participants{Kheireddine Berkane,
              Pierre-luc Blot,
              Majid Tasserie, 
              Ahouate Abdellatif,
         	  Nasser Adjibi}
\absents{Idrissou Amzath,
             Alexandre Petre}
\filiere{M1GIL}
\projet{Compilateur LLVM}
\projdesc{Langage jouet KAWA} % si la description est trop longue pensez 
                              % à mettre \\ pour un retour à la ligne
\title{Compte rendu du Mercredi 21 Janvier 2015}
\version{n°1}

%\histentry{0.5}{10/12/2012}{Changelog pour la d\'emo}

\begin{document}
    \maketitle
    \tableofcontents
    \clearpage
    
    \section{Observations et questions} 
    \label{sec:observations_et_questions}
	On s'est mis d'accord sur un plan de recherche qui correspond à la 1ére tâche de plan de développement.
        \begin{itemize}
            \item  On s'est dévisé on 2 sous-équipes :
		\begin{itemize}
			\item Sous-Equipe A: Abdellatif et Majid.
			\item Sous-Equipe B: Pierre-luc, kheireddine, Nasser, Alexendre et Hamzath		
		\end{itemize}
	    \item Objectifs:
		\begin{itemize}
			\item Sous-Equipe A: Front END (Analyse Lexicale, analyse syntaxique et l'analyse sémantique).
			\item Sous-Equipe B: Passage de la sémantique au Code IR (LLVM).		
		\end{itemize}
        \end{itemize}
        
\end{document}
