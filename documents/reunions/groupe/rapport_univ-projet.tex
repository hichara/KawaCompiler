\documentclass{../../res/rapport}

\usepackage[T1]{fontenc}
%\usepackage[utf8]{inputenc}
%\usepackage{DejaVuSans}

\logo{../../res/logo_univ.png}
\author{IDRISSOU Amzath}
\participants{Kheireddine Berkane,
              Pierre-luc Blot,
              Nasser Adjibi , 
              Ahouate Abdellatif,
              Idrissou Amzath}
\absents{Alexandre Petre,
         Madjid Tasserie}
\filiere{M1GIL}
\projet{Compilateur LLVM}
\projdesc{Langage jouet KAWA} % si la description est trop longue pensez 
                              % à mettre \\ pour un retour à la ligne
\title{Compte rendu du Vendredi 14 novembre 2014}
\version{n°2}

%\histentry{0.5}{10/12/2012}{Changelog pour la d\'emo}

\begin{document}
    \maketitle
    \tableofcontents
    \clearpage
    
    \section{Observations et questions} 
    \label{sec:observations_et_questions}
        \begin{itemize}
            \item Nous avons revue notre diagramme de cas d’utilisation pour y 
                ajouter les cas manquants L’attribution des rôles à chaque
                membre du groupe.\\

                \begin{itemize}
                    \item Responsable client : Kheireddine
                    \item Responsable Technique : Nasser, Abdellatif ,Madjid
                    \item Testeur : Amzath, Allexandre
                    \item Chef de projet : Pierre-Luc  
                \end{itemize}
            \item  L’attribution de la rédaction des documents:
                \begin{itemize}
                    \item STB: Nasser et Kheireddine
                    \item DAL: Nasser et Kheireddine
                    \item PDD: Pierre-Luc et Madjid
                    \item CDR: Allexandre et Amzath 
                    \item ADR: Pierre-Luc et Madjid        
                \end{itemize}
            
        \end{itemize}
        
    \section{Plan d'actions} 
    \label{sec:plan_d_actions}
        \begin{itemize}
            \item Prochaine réunion en groupe est pour vendredi prochain
                10h15-12h15
            \item Trouver une solution pour compiler avec la version LLVM 3.5. 
            \item Revoir le document STB pour apporter les modifications afin
                de le valider par le client la réunion de jeudi prochain.
            \item Valider l’approche proposée par le client concernant le 
                diagramme des cas d’utilisations par nos professeurs spécialisés
                dans l’UML.
        \end{itemize}
        
\end{document}
