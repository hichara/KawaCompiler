\documentclass{../res/univ-projet}

%Import des packages utilisés pour le document
\usepackage[utf8x]{inputenc}
\usepackage[francais]{babel}
\usepackage[T1]{fontenc}
%\usepackage{array}
%\usepackage{hyperref}
%\usepackage{tabularx, longtable}
%\usepackage[table]{xcolor}
%\usepackage{fancyhdr}
%\usepackage{lastpage}

\definecolor{gris}{rgb}{0.95, 0.95, 0.95}

%Redéfinition des marges
%\addtolength{\hoffset}{-2cm}
%\addtolength{\textwidth}{4cm}
\addtolength{\topmargin}{-1cm}
\addtolength{\textheight}{1cm}
\addtolength{\headsep}{0.8cm} 
\addtolength{\footskip}{-0.2cm}


%Import page de garde et structures pour la gestion de projet
%\usepackage{structures}

%Variables
\logo{../res/logo_univ.png}
\title{Plan de développement}
\author{Pierre-Luc BLOT, Kheireddine \bsc{Berkane}}
\projet{Compilateur LLVM}
\projdesc{Langage jouet Kawa}
\filiere{M1GIL - Conduite de Projet}
\version{0.1}
\relecteur{Nasser \bsc{ADJIBI}}
%\signataire{Florent \bsc{NICART}}
\date{\today}

\histentry{0.1}{23/12/2014}{Version initiale.}


% -- Début du document -- %
\begin{document}

%Page de garde
\maketitle
\newpage
%La table des matières
\tableofcontents
\newpage

\section{Objet}

% Présentation succinte du sujet et hyp de travail.
LLVM est une infrastructure modulaire permettant la réalisation de
chaînes de compilation et conçue pour l'optimisation. Elle met en oeuvre
une représentation intermédiaire du code qui permet de découpler les
langages de l'architecture. Nore objectif est de réaliser, à l'aide de
l'infrastructure LLVM, un comilateur pour un langage jouet, que nous
appellerons Kawa, ce dernier doit supporter:
 \begin{itemize}
 	\item	Les classes, les classes abstraites, les interfaces
	\item   L'héritage
	\item   Le polymorphisme
	\item   Le système de types sera composé des types primitifs
	  (int, float, etc.), des classes et des interfaces
	\item   Les instructions de contrôle telles ques (if/else, for,
	  while/do, switch, etc.).
	\item   Les méthodes seront définies de manière identique à Java
	  execpté que les paramètres pourront être préfixés du mot clé
	  \bsc{value} (transmission par valeur au lieu de référence)
 \end{itemize}


\subsection{Besoins opérationnels}
\subsection{Objectifs techniques}
\subsection{Contraintes et recommendations}
\subsection{Résultats attendus}

\section{Documents applicables et de référence}
% Liste des
% - Références des documents quidefinissent formellement les principes
%   directeurs et le hypothèse de travail prise en compte pour l'établissement de la spécification.
% - Références des documents cités dans la STB au titre d'explication ou de justification.
Différents documents de référence :
\begin{itemize}

\item Le site LLVM \href{http://llvm.org}{llvm.org}.
\item Le document de spécification du client \href{file:../client/spec1.pdf}{Spec1.pdf}
\end{itemize}

\section{Terminologie et sigles utilisés}
  \begin{itemize}
  	\item LLVM : (Low Level Virtual Machine) est une infrastructure de compilateur conçue optimisation à la compilation.
	\item Kawa : langage jouet qui reprend quelques fonctionnalités de java.
	\item Clang : compilateur pour le langage c++, son interface de bas niveau  se base sur des bibliothèques llvm pour la compilation. Il est utilisé par APPELE.
	\item GIT : logiciel qui permet de stocker un ensemble de fichiers en conservant la chronologie de toutes les modifications qui ont été effectuées dessus.
	\item ELF : (Executable and Linkable Format) est un format de fichier binaire utilisé pour l'enregistrement de code compilé (objets, exécutables, bibliothèques de fonctions).
	\item Ubuntu : Distribution linux sur base Debian.
	\item C++ : Langage de programmation orienté objet bas niveau.
	\item POO : Programmation orientée objet.
	\item Makefile : Fichier regroupant des instructions de compilation avec gestion de dépendances.
  \end{itemize}

\section{Exigences fonctionnelles}
\subsection{Présentation de la mission du produit logiciel}

\begin{tabular}{|>{\centering}p{1cm}|>{\centering}p{7cm}|>{\centering}p{2.5cm}|>{\centering}p{3cm}|}
  \hline
  \color{white}\cellcolor{blue}\bfseries{Id}&
  \color{white}\cellcolor{blue}\bfseries{Intitulé}&
  \color{white}\cellcolor{blue}\bfseries{Acteur(s)}&
  \color{white}\cellcolor{blue}\bfseries{Priorité}\\
  \cr
  \hline EF\_1&
  Afficher l'aide&
  Utilisateur&
  Indispensable
  \cr
  \hline EF\_2&
  Compiler une application en mode monolithique &
   Utilisateur&
  Indispensable
  \cr
  \hline EF\_3&
  Compiler une application en mode partagé& 
  Utilisateur&
  Secondaire
  \cr
  \hline EF\_4&
  Compiler application utilisant des bibliothèques partagées&
  Utilisateur&
  Important
  \cr
  \hline EF\_5&
  Afficher la version du compilateur&
  Utilisateur&
  Important
  \cr
  \hline EF\_6&
  Indiquer les chemins des dépendances entre sources et classes déjà compilées&
  Utilisateur&
  Important
  \cr
  \hline EF\_7&
  Activer l'affichage en couleur&
  Utilisateur&
  Secondaire
  \cr
  
  \hline
\end{tabular}\\

\newpage

\begin{figure}
  \centering
  \includegraphics[scale=0.8]{../res/stb/vs_finale_usecase.jpg}
  \caption{\textbf{Cas d'utilisations du compilateur kawa.}}
\end{figure}

%Cas d'utilisation
\subsection{Cas d'utilisation EF\_1}
\fiche
{Afficher l'aide}                    % Nom du cas d'utilisation
{Utilisateur du compilateur}                               % Acteurs concernés
{                                                % Description
  le compilateur affiche 
   la liste des options du compilateur sur la sortie standard à travers une ligne de commande.
}
{
  L'odre de priorité entre le commutateur de help (-h ou --help) et le commutateur de version (-v ou --version) est définit par la première occurrence de l'un des deux commutateur i-e si nous avons un \textbf {-h} avant un \textbf {-v}, le compilateur annule tout le reste et affiche le help 
}                                                % Préconditions
{Commutateur de ligne de commande :kawac -h ou --help}                             % Evénements déclenchants
{Action utilisateur permettant de quiter ce mode }                       % Conditions d'arrêt
% {0.6}{../res/stb/usecase2_flot_event.png}      % Diagramme
{} % acteur(s)
{} % system
{} % flot exceptionsb
 % fin usecase EF_1

\subsection{Cas d'utilisation EF\_2}
\fiche
{Compiler une application en mode monolithique}                    % Nom du cas d'utilisation
{Utilisateur du compilateur}                               % Acteurs concernés
{                                                % Description
  L'utilisateur introduit un ensemble de classes
  kawa afin de les précompiler et de générer le tous dans un seul exécutable.
}
{
	Ensemble de fichiers sources 
	respectant la syntaxe du langage kawa.La non présence des deux commutateurs (-h ou --help) et (-v ou --version) dans la ligne de commande permettant la compilation est obligatoire
}                                                % Préconditions
{Commutateur de ligne de commande :kawac -m filessources.}                             % Evénements déclenchants
{Fin du programme (succès de la compilation),ou bien un message renvoyé par le compialteur indiquant une erreur rencontrée lors de l'analyse de programme, ainsi que dans certains cas le compilateur ne touve pas les fichiers à compiler. 
}                       % Conditions d'arrêt
%{0.6}{../res/stb/usecase2_flot_event.png} 		 % Diagramme
{                                                % Flots d'exceptions
  
}
{} % system
{Abondant provoqué par l'utilisateur, comme la fermeture du terminal au cours de la compilation. } % flot exceptions
% Fin de la fiche du cas d'utilisation 1.


%Cas d'utilisation 2
\subsection{Cas d'utilisation EF\_3}
\fiche
{Compiler une application en mode partagé}                    % Nom du cas d'utilisation
{Utilisateur du compilateur}                               % Acteurs concernés
{                                                % Description
  L’utilisateur introduit un ensemble de sources kawa
 qui peuvent appelés des bibliothèques externes, le compilateur doit être capable de chercher les bibliothèques pour les utiliser ou bien  de les recompiler si nécessaire.La distinction entre ce mode et le mode monolithique c'est l'absence du commutateur -m 
}
{
	Ensemble de fichiers source respectant la syntaxe du langage kawa qui peuvent utiliser des bibliothèques partagées.La non présence des deux commutateurs (-h ou --help) et (-v ou --version) dans la ligne de commande permettant la compilation est obligatoire
	
}                                                % Préconditions
{Commutateur de ligne de commande:kawac filessources [options]}                             % Evénements déclenchants
{Fin du programme (succès de la compilation),ou bien un message renvoyé par le compialteur indiquant une erreur rencontrée lors de l'analyse de programme, ainsi que dans certains cas le compilateur ne touve pas les fichiers à compiler.} % Conditions d'arrêt
%{0.6}{../res/stb/usecase2_flot_event.png} 		 % Diagramme
{                                                % Flots d'exceptions
  
}{} % system
{Abondant provoqué par l'utilisateur,comme la fermeture du terminal au cours de la compilation.} % flot exceptions
% Fin de la fiche du cas d'utilisation 2.


%Cas d'utilisation
\subsection{Cas d'utilisation EF\_4}
\fiche
{Compiler application utilisant des bibliothèques partagées}                      % Nom du cas d'utilisation
{Utilisateur du compilateur}                               % Acteurs concernés
{                                                % Description
    L'utilisateur peut compiler des bibliothèques dynamiques en compilant son application obligatoirement dans un mode partagé,et ceci est possible même si son source n'utilise pas forcément ces bibliothèques externes.      
}
{
  La non présence des deux commutateurs (-h ou --help) et (-v ou --version) dans la ligne de commande permettant la compilation est obligatoire.
}                                                % Préconditions
{Commutateur de ligne de commande:kawac filessources [options] } % Evénements déclenchants
{Fin du programme (succès de la compilation),ou bien un message renvoyé par le compialteur indiquant une erreur rencontrée lors de l'analyse de programme, ainsi que dans certains cas le compilateur ne touve pas les fichiers à compiler.} % Conditions d'arrêt
%{0.6}{../res/stb/usecase3_flot_event.png}     % Diagramme
{                                                % Flots d'exceptions
 
}{} % system
{Abondant provoqué par l'utilisateur,comme la fermeture du terminal au cours de la compilation.} % flot exceptions
% Fin de la fiche du cas d'utilisation 3.
%Cas d'utilisation
\subsection{Cas d'utilisation EF\_5}
\fiche
{Afficher la version du compilateur}                      % Nom du cas d'utilisation
{Utilisateur du compilateur}                               % Acteurs concernés
{                                                % Description
   
L'utilisateur peut savoir la version du compilateur avec le quel compile ses sources et ses bibliothèques  à travers un commutateur de ligne de commande.   
}
{
   L'odre de priorité entre le commutateur de help (-h ou --help) et le commutateur de version (-v ou --version) est définit par la première occurrence de l'un des deux commutateur i-e si nous avons un \textbf {-v} avant un \textbf {-h}, le compilateur annule tout le reste et affiche la version
}                                                % Préconditions
{Commutateur de ligne de commande:kawac -v ou --version}                             % Evénements déclenchants
{Fin du programme.}                       % Conditions d'arrêt
%{0.6}{../res/stb/usecase3_flot_event.png}     % Diagramme
{                                                % Flots d'exceptions
 
}{} % system
{} % flot exceptions
% Fin de la fiche du cas d'utilisation 3.
%Cas d'utilisation
\subsection{Cas d'utilisation EF\_6}
\fiche
{Indiquer les chemins des dépendances entre sources et classes déjà compilées}          % Nom du cas d'utilisation
{Utilisateur du compilateur}                               % Acteurs concernés
{                                                % Description
   
L'utilisateur peut définir les dépendances pour la compilation de son application, en indiquant des chemins entre les sources et des fichiers déjà compilés.}
{
  
}                                                % Préconditions
{Commutateur de ligne de commande:kawac filessources -d path}                             % Evénements déclenchants
{Fin du programme (succès de la compilation),ou bien un message renvoyé par le compialteur indiquant une erreur rencontrée lors de l'analyse de programme, ainsi que dans certains cas le compilateur ne touve pas les fichiers à compiler.}                       % Conditions d'arrêt
%{0.6}{../res/stb/usecase3_flot_event.png}     % Diagramme
{                                                % Flots d'exceptions
 
}{} % condition d'arret
{Abondant provoqué par l'utilisateur, comme la fermeture du terminal au cours de la compilation.} % flot exceptions
% Fin de la fiche du cas d'utilisation 3.

%Cas d'utilisation
\subsection{Cas d'utilisation EF\_7}
\fiche
{Activer l'affichage en couleur}          % Nom du cas d'utilisation
{Utilisateur du compilateur}                               % Acteurs concernés
{                                                % Description
   
  L'utilisateur peut activer l'option de l'affichage en couleur, afin de décorer les messages renvoyés par le compilateur dans les différents modes de compilation.   
}
{
  
}                                                % Préconditions
{Commutateur de ligne de commande:kawac filessource --color}                             % Evénements déclenchants
{Fin du programme.}                       % Conditions d'arrêt
%{0.6}{../res/stb/usecase3_flot_event.png}     % Diagramme
{                                                % Flots d'exceptions
 
}{} % system
{} % flot exceptions
% Fin de la fiche du cas d'utilisation 3.

\section{Exigences opérationnelles}
\begin{tabular}{|>{\centering}p{1,5cm}|>{\centering}p{10cm}|>{\centering}p{3cm}|}
  \hline
  \color{white}\cellcolor{blue}\bfseries{Id}&
  \color{white}\cellcolor{blue}\bfseries{Intitulé}&
  \color{white}\cellcolor{blue}\bfseries{Priorité}\\
  \cr
  \hline
\end{tabular}\\

\section{Exigences opérationnelles d'interface}

\begin{tabular}{|>{\centering}p{1,5cm}|>{\centering}p{10cm}|>{\centering}p{3cm}|}
  \hline
  \color{white}\cellcolor{blue}\bfseries{Id}&
  \color{white}\cellcolor{blue}\bfseries{Intitulé}&
  \color{white}\cellcolor{blue}\bfseries{Priorité}\\
  \cr
  \hline
\end{tabular}\\


\section{Exigences de qualité}

\begin{tabular}{|>{\centering}p{1,5cm}|>{\centering}p{10cm}|>{\centering}p{3cm}|}
  \hline
  \color{white}\cellcolor{blue}\bfseries{Id}&
  \color{white}\cellcolor{blue}\bfseries{Intitulé}&
  \color{white}\cellcolor{blue}\bfseries{Priorité}\\
  \cr
  \hline
  EQ\_1&
  les messages d’erreurs
renvoyés par le compilateur doivent être explicites, les plus fines possibles et
surtout par rapport à ce qu'il s'est passé.&
  Important
  \cr
  \hline
\end{tabular}\\

\section{Exigences de réalisation}

\begin{tabular}{|>{\centering}p{1,5cm}|>{\centering}p{10cm}|>{\centering}p{3cm}|}
  \hline
  \color{white}\cellcolor{blue}\bfseries{Id}&
  \color{white}\cellcolor{blue}\bfseries{Intitulé}&
  \color{white}\cellcolor{blue}\bfseries{Priorité}\\
  \cr
  \hline
  ER\_1&
  La compilation produit un exécutable linux au format ELF&
  Indispensable
  \cr
  \hline
  ER\_2&
  Pas d'implémentation des cast&
  \cr
  \hline
  ER\_3&
  Pas d'implémentation de la générécité&
  \cr
  \hline
  ER\_4&
  Implémentation en code monolithique&
  Indispensable
  \cr
  \hline
  ER\_5&
  Implémentation en code partagé&
  Optionnel
  \cr
  \hline
  ER\_6&
  Pas d'attributs dans les interfaces. Seulement des méthodes.
  \cr
  \hline
\end{tabular}\\

\end{document}

