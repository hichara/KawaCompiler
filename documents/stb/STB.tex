\documentclass{../res/univ-projet}

%Import des packages utilisés pour le document
\usepackage[utf8x]{inputenc}
\usepackage[francais]{babel}
\usepackage[T1]{fontenc}
%\usepackage{array}
%\usepackage{hyperref}
%\usepackage{tabularx, longtable}
%\usepackage[table]{xcolor}
%\usepackage{fancyhdr}
%\usepackage{lastpage}

\definecolor{gris}{rgb}{0.95, 0.95, 0.95}

%Redéfinition des marges
%\addtolength{\hoffset}{-2cm}
%\addtolength{\textwidth}{4cm}
\addtolength{\topmargin}{-1cm}
\addtolength{\textheight}{1cm}
\addtolength{\headsep}{0.8cm} 
\addtolength{\footskip}{-0.2cm}


%Import page de garde et structures pour la gestion de projet
%\usepackage{structures}

%Variables
\logo{../res/logo_univ.png}
\title{Spécification Technique de Besoin}
\author{Kheireddine \bsc{Berkane}, Pierre-Luc \bsc{BLOT}}
\projet{Compilateur LLVM}
\projdesc{Langage jouet Kawa}
\filiere{M1GIL - Conduite de Projet}
\version{0.1}
%\relecteur{Pierre-Luc \bsc{BLOT}}
%\signataire{Florent \bsc{NICART}}
\date{\today}

\histentry{0.1}{04/11/2014}{Version initiale.}

% -- Début du document -- %
\begin{document}

%Page de garde
\maketitle
\newpage
%La table des matières
\tableofcontents
\newpage

\section{Objet}

% Présentation succinte du sujet et hyp de travail.
LLVM est une infrastructure modulaire permettant la réalisation de
chaînes de compilation et conçue pour l'optimisation. Elle met en oeuvre
une représentation intermédiaire du code qui permet de découpler les
langages de l'architecture. Nore objectif est de réaliser, à l'aide de
l'infrastructure LLVM, un comilateur pour un langage jouet, que nous
appellerons Kawa, ce dernier doit supporter:
 \begin{itemize}
 	\item	Les classes, les classes abstraites, les interfaces
	\item   L'héritage
	\item   Le polymorphisme
	\item   Le système de types sera composé des types primitifs
	  (int, float, etc.), des classes et des interfaces
	\item   Les instructions de contrôle telles ques (if/else, for,
	  while/do, switch, etc.).
	\item   Les méthodes seront définies de manière identique à Java
	  execpté que les paramètres pourront être préfixés du mot clé
	  \bsc{value} (transmission par valeur au lieu de référence)
 \end{itemize}


\subsection{Besoins opérationnels}
\subsection{Objectifs techniques}
\subsection{Contraintes et recommendations}
\subsection{Résultats attendus}

\section{Documents applicables et de référence}
% Liste des
% - Références des documents quidefinissent formellement les principes
%   directeurs et le hypothèse de travail prise en compte pour l'établissement de la spécification.
% - Références des documents cités dans la STB au titre d'explication ou de justification.
Différents documents de référence :
\begin{itemize}

\item Le site LLVM \href{http://llvm.org}{llvm.org}.
\item Le document de spécification du client \href{file:../client/spec1.pdf}{Spec1.pdf}
\end{itemize}

\section{Terminologie et sigles utilisés}
\textcolor{blue}{
  \begin{itemize}
  	\item LLVM : (Low Level Virtual Machine) est une infrastructure de compilateur conçue optimisation à la compilation.
	\item Kawa : langage jouet qui reprend quelques fonctionnalités de java.
	\item Clang : compilateur pour le langage c++, son interface de bas niveau  se base sur des bibliothèques llvm pour la compilation. Il est utilisé par APPELE.
	\item GIT : logiciel qui permet de stocker un ensemble de fichiers en conservant la chronologie de toutes les modifications qui ont été effectuées dessus.
	\item ELF : (Executable and Linkable Format) est un format de fichier binaire utilisé pour l'enregistrement de code compilé (objets, exécutables, bibliothèques de fonctions).
	\item Ubuntu : Distribution linux sur base Debian.
	\item C++ : Langage de programmation orienté objet bas niveau.
	\item POO : Programmation orientée objet.
	\item Makefile : Fichier regroupant des instructions de compilation avec gestion de dépendances.
  \end{itemize}
}

\section{Exigences fonctionnelles}
\subsection{Présentation de la mission du produit logiciel}

\begin{tabular}{|>{\centering}p{1cm}|>{\centering}p{7cm}|>{\centering}p{2.5cm}|>{\centering}p{3cm}|}
  \hline
  \color{white}\cellcolor{blue}\bfseries{Id}&
  \color{white}\cellcolor{blue}\bfseries{Intitulé}&
  \color{white}\cellcolor{blue}\bfseries{Acteur(s)}&
  \color{white}\cellcolor{blue}\bfseries{Priorité}\\
  \cr
  \hline EF\_1&
  Compilation code source Kawa&
  Développeur Kawa&
  Indispensable
  \cr
  \hline EF\_2&
  Compilation classe&
  Compilateur&
  Indispensable
  \cr
  \hline EF\_3&
  Compilation classe abstraite& 
  Compilateur&
  Important
  \cr
  \hline EF\_4&
  Compilation interface&
  Compilateur&
  Important
  \cr
  \hline EF\_5&
  Compiler notion héritage&
  Compilateur&
  Important
  \cr
  \hline EF\_6&
  Compiler notion polymorphisme&
  Compilateur&
  Important
  \cr
  \hline EF\_7&
  Compiler notion polymorphisme ad hoc&
  Compilateur&
  Important
  \cr
  \hline EF\_8&
  Compiler notion polymorphisme de type&
  Compilateur&
  Important
  \cr
  \hline EF\_9&
  Compiler bloc de controle&
  Compilateur&
  Important
  \cr
  \hline EF\_10&
  Compiler méthodes&
  Compilateur&
  Important
  \cr
  \hline EF\_11&
  Compiler méthodes avec paramètres préfixés d'un mot clé \bsc{value}&
  Compilateur&
  Important
  \cr
  \hline EF\_12&
  Compiler système de type&
  Compilateur&
  Important
  \cr
  \hline EF\_13&
  Compiler affectation&
  Compilateur& Indispensable
  \cr
  \hline EF\_14&
  Compiler opérateurs arithmétiques sur type primitif&
  Compilateur& Important
  \cr
  \hline
\end{tabular}\\

%\newpage

%Cas d'utilisation
\subsection{Cas d'utilisation 1}
\ficheGraphic
{Compilation de classes kawa}                    % Nom du cas d'utilisation
{Développeur Kawa}                               % Acteurs concernés
{                                                % Description
  Le développeur introduit un ensemble de classes
  kawa afin de les précompiler.
}
{
	Ensemble de fichiers source de classes
	respectant la syntaxe du langage kawa. 
}                                                % Préconditions
{Ligne de commande.}                             % Evénements déclenchants
{Erreur (syntaxe, fichier introuvable, ..), ou 
 succès de la compilation}                       % Conditions d'arrêt
{0.6}{../res/stb/usecase1_flot_event.png} 		 % Diagramme
{                                                % Flots d'exceptions
  \begin{itemize}
  \item Abandon de l'utilisateur.
  \end{itemize}
}
% Fin de la fiche du cas d'utilisation 1.


%Cas d'utilisation 2
\subsection{Cas d'utilisation 2}
\ficheGraphic
{Compilation d'interface kawa}                    % Nom du cas d'utilisation
{Développeur Kawa}                               % Acteurs concernés
{                                                % Description
  Le développeur introduit un ensemble d'interfaces
  kawa afin de les précompiler.
}
{
	Ensemble de fichiers source d'interfaces
	respectant la syntaxe du langage kawa. 
}                                                % Préconditions
{Ligne de commande.}                             % Evénements déclenchants
{Erreur (syntaxe, fichier introuvable, ..), ou 
 succès de la compilation}                       % Conditions d'arrêt
{0.6}{../res/stb/usecase2_flot_event.png} 		 % Diagramme
{                                                % Flots d'exceptions
  \begin{itemize}
  \item Abandon de l'utilisateur.
  \end{itemize}
}
% Fin de la fiche du cas d'utilisation 2.


%Cas d'utilisation
\subsection{Cas d'utilisation 3}
\ficheGraphic
{Compilation d'exécutable}                      % Nom du cas d'utilisation
{Développeur Kawa}                               % Acteurs concernés
{                                                % Description
  Le développeur introduit un ensemble de classes
  et interfaces kawa précompiler afin de
  créer un fichier exécutable.
}
{
	Ensemble de fichiers précompilés.
}                                                % Préconditions
{Ligne de commande.}                             % Evénements déclenchants
{Erreur (fichier introuvable, ..), ou 
 succès de la compilation}                       % Conditions d'arrêt
{0.6}{../res/stb/usecase3_flot_event.png} 		 % Diagramme
{                                                % Flots d'exceptions
  \begin{itemize}
  \item Abandon de l'utilisateur.
  \end{itemize}
}
% Fin de la fiche du cas d'utilisation 3.
\vspace{0.5cm}

\section{Exigences opérationnelles}
\begin{tabular}{|>{\centering}p{1,5cm}|>{\centering}p{10cm}|>{\centering}p{3cm}|}
  \hline
  \color{white}\cellcolor{blue}\bfseries{Id}&
  \color{white}\cellcolor{blue}\bfseries{Intitulé}&
  \color{white}\cellcolor{blue}\bfseries{Priorité}\\
  \cr
  \hline
\end{tabular}\\

\section{Exigences opérationnelles d'interface}

\begin{tabular}{|>{\centering}p{1,5cm}|>{\centering}p{10cm}|>{\centering}p{3cm}|}
  \hline
  \color{white}\cellcolor{blue}\bfseries{Id}&
  \color{white}\cellcolor{blue}\bfseries{Intitulé}&
  \color{white}\cellcolor{blue}\bfseries{Priorité}\\
  \cr
  \hline
\end{tabular}\\


\section{Exigences de qualité}

\begin{tabular}{|>{\centering}p{1,5cm}|>{\centering}p{10cm}|>{\centering}p{3cm}|}
  \hline
  \color{white}\cellcolor{blue}\bfseries{Id}&
  \color{white}\cellcolor{blue}\bfseries{Intitulé}&
  \color{white}\cellcolor{blue}\bfseries{Priorité}\\
  \cr
  \hline
  EQ\_1&
  Le compilateur ne plante pas et affiche des messages d'erreurs clairs&
  Important
  \cr
  \hline
\end{tabular}\\

\section{Exigences de réalisation}

\begin{tabular}{|>{\centering}p{1,5cm}|>{\centering}p{10cm}|>{\centering}p{3cm}|}
  \hline
  \color{white}\cellcolor{blue}\bfseries{Id}&
  \color{white}\cellcolor{blue}\bfseries{Intitulé}&
  \color{white}\cellcolor{blue}\bfseries{Priorité}\\
  \cr
  \hline
  ER\_1&
  La compilation produit un exécutable linux au format ELF&
  Indispensable
  \cr
  \hline
  ER\_2&
  Pas d'implémentation des cast&
  \cr
  \hline
  ER\_3&
  Pas d'implémentation de la générécité&
  \cr
  \hline
  ER\_4&
  Implémentation en code monolithique&
  Indispensable
  \cr
  \hline
  ER\_5&
  Implémentation en code partagé&
  Optionnel
  \cr
  \hline
  ER\_6&
  Pas d'attributs dans les interfaces. Seulement des méthodes.
  \cr
  \hline
\end{tabular}\\

\end{document}

