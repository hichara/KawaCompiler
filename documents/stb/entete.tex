\title{Spécification Technique de Besoin}
\author{Kheireddine \bsc{Berkane}, Nasser \bsc{Adjibi}}
\projet{Compilateur LLVM}
\projdesc{Langage jouet Kawa}
\filiere{M1GIL - Conduite de Projet}
\version{0.7}
\relecteur{Pierre-Luc \bsc{BLOT}, Alexandre \bsc{PETRE}}
%\signataire{Florent \bsc{NICART}}
\date{\today}

\histentry{0.6}{11/04/2015}{ 
\begin{itemize}
\item Élimination de l'exigence fonctionnelle EF\_3 : Compiler une application en mode partagé 
\item Élimination de l'exigence fonctionnelle EF\_4 : Compiler une bibliothèques partagée
\item Élimination de l'exigence fonctionnelle EF\_6 : Indiquer les chemins des dépendances entre le source de l'application et des modules (classe/interface) externes
\item Élimination de l'exigence de réalisation EXR\_4 : Compilation d'application partagée
\item Élimination de l'exigence de réalisation EXR\_6 : Compilation de bibliothèque partagée
\item Élimination de l'exigence de réalisation EXR\_25 : Définition d'attributs ou de variables de type \textbf{value}
\item Élimination de l'exigence de réalisation EXR\_26 : Définition méthode value
\item Élimination de l'exigence de réalisation EXR\_41 :  Gestion des exceptions
\item Changement du diagramme de cas d'utilisation après la discussion avec client afin de réduire quelques exigences fonctionnelles et de réalisation 
\end{itemize}
}
\histentry{0.7}{15/04/2015}{ 
\begin{itemize}
\item Mettre l'exigence fonctionnelle EF\_3 : Afficher la version du compilateur en secondaire
\item Élimination de l'exigence fonctionnelle EF\_4 :Activer l'affichage en couleur
\item Élimination de l'exigence de réalisation EXR\_5 : Garbage collector
\end{itemize}
}
