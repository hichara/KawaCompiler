\title{Spécification Technique de Besoin}
\author{Kheireddine \bsc{Berkane}, Nasser \bsc{Adjibi}}
\projet{Compilateur LLVM}
\projdesc{Langage jouet Kawa}
\filiere{M1GIL - Conduite de Projet}
\version{0.7}
\relecteur{Pierre-Luc \bsc{BLOT}, Alexandre \bsc{PETRE}}
%\signataire{Florent \bsc{NICART}}
\date{\today}

\histentry{0.1}{04/11/2014}{Version initiale.}
\histentry{0.1.5}{18/11/2014}{Définition des cas d'utilisations et des exigences.}
\histentry{0.2}{03/12/2014}{Modifications par rapport au retour client du 25/11/2014.}
\histentry{0.3}{10/12/2014}{ \begin{itemize}
								 \item Les parties événements déclenchants ont été détaillées.
								 \item Les parties flots d'exceptions, ainsi que les conditions d’arrêts de toutes les exigences fonctionnelles ont été détaillées.
								 \item L'ajout des exigences opérationnelles d'interface.
								 \item Correction des erreurs signalées lors de la réunion client 04/12/2014.	
							 \end{itemize}	 	
								 }
\histentry{0.4}{22/12/2014}{ \begin{itemize}
\item Les parties événements déclenchants ont été modifiées et structurées sous forme d'une liste.
\item Les parties flots d'exceptions, ainsi que les conditions d’arrêts de toutes les exigences fonctionnelles ont été été modifiées et structurées sous forme d'une liste.
\item La suppression des exigences opérationnelles d'interface après une remarque faite par le prof du TP gestion de projet.
\item Changement de priorité de l'exigence fonctionnelle EF\_4 en secondaire car elle dépend de l'exigence EF\_3 qui est secondaire.
\item Correction des erreurs signalées lors du retour client par mail le 19/12/2014.	
\end{itemize}	 	
}
\histentry{0.5}{18/01/2015}{ \begin{itemize}
\item Élimination des exigences de réalisations qui correspondent aux exigences fonctionnelles afin d'éviter les redondances inutiles. 
\item La partie des exigences fonctionnelles a été détaillée ainsi que les scénarios des exceptions ont été spécifiés afin de faciliter les tests après.
\item Modifications apportées par rapport à la revue du lancement du projet 19/01/2015.
\end{itemize}
}
\histentry{0.6}{11/04/2015}{ 
\begin{itemize}
\item Élimination de l'exigence fonctionnelle EF\_3 : Compiler une application en mode partagé 
\item Élimination de l'exigence fonctionnelle EF\_4 : Compiler une bibliothèques partagée
\item Élimination de l'exigence fonctionnelle EF\_6 : Indiquer les chemins des dépendances entre le source de l'application et des modules (classe/interface) externes
\item Élimination de l'exigence de réalisation EXR\_4 : Compilation d'application partagée
\item Élimination de l'exigence de réalisation EXR\_6 : Compilation de bibliothèque partagée
\item Élimination de l'exigence de réalisation EXR\_25 : Définition d'attributs ou de variables de type \textbf{value}
\item Élimination de l'exigence de réalisation EXR\_26 : Définition méthode value
\item Élimination de l'exigence de réalisation EXR\_41 :  Gestion des exceptions
\item Changement du diagramme de cas d'utilisation après la discussion avec client afin de réduire quelques exigences fonctionnelles et de réalisation 
\end{itemize}
}

\newpage
\histentry{0.7}{15/04/2015}{ 
\begin{itemize}
\item Mettre l'exigence fonctionnelle EF\_3 : Afficher la version du compilateur en secondaire
\item Élimination de l'exigence fonctionnelle EF\_4 :Activer l'affichage en couleur
\item Élimination de l'exigence de réalisation EXR\_5 : Garbage collector
\end{itemize}
}
