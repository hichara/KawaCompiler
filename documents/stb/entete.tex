\title{Spécification Technique de Besoin}
\author{Kheireddine \bsc{Berkane}, Nasser \bsc{Adjibi}}
\projet{Compilateur LLVM}
\projdesc{Langage jouet Kawa}
\filiere{M1GIL - Conduite de Projet}
\version{0.4}
\relecteur{Pierre-Luc \bsc{BLOT}}
%\signataire{Florent \bsc{NICART}}
\date{\today}

\histentry{0.1}{04/11/2014}{Version initiale.}
\histentry{0.1.5}{18/11/2014}{Définition des cas d'utilisations et des exigences.}
\histentry{0.2}{03/12/2014}{Modifications par rapport au retour client du 25/11/2014.}
\histentry{0.3}{10/12/2014}{ \begin{itemize}
								 \item Les parties événements déclenchants ont été détaillées.
								 \item Les parties flots d'exceptions, ainsi que les conditions d’arrêts de toutes les exigences fonctionnelles ont été détaillées.
								 \item L'ajout des exigences opérationnelles d'interface.
								 \item Correction des erreurs signalées lors de la réunion client 04/12/2014.	
							 \end{itemize}	 	
								 }
\histentry{0.4}{22/12/2014}{ \begin{itemize}
\item Les parties événements déclenchants ont été modifiées et structurées sous forme d'une liste.
\item Les parties flots d'exceptions, ainsi que les conditions d’arrêts de toutes les exigences fonctionnelles ont été été modifiées et structurées sous forme d'une liste.
\item La suppression des exigences opérationnelles d'interface après une remarque faite par le prof du TP gestion de projet.
\item Changement de priorité de l'exigence fonctionnelle EF\_4 en secondaire car elle dépend de l'exigence EF\_3 qui est secondaire.
\item Correction des erreurs signalées lors du retour client par mail le 19/12/2014.	
\end{itemize}	 	
}