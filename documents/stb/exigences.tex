\section{Exigences opérationnelles}
\begin{tabular}{|>{\centering}p{1,5cm}|>{\centering}p{10cm}|>{\centering}p{3cm}|}
  \hline
  \color{white}\cellcolor{blue}\bfseries{Id}&
  \color{white}\cellcolor{blue}\bfseries{Intitulé}&
  \color{white}\cellcolor{blue}\bfseries{Priorité}\\
  \cr
  \hline
\end{tabular}\\

\section{Exigences opérationnelles d'interface}

\begin{tabular}{|>{\centering}p{1,5cm}|>{\centering}p{10cm}|>{\centering}p{3cm}|}
  \hline
  \color{white}\cellcolor{blue}\bfseries{Id}&
  \color{white}\cellcolor{blue}\bfseries{Intitulé}&
  \color{white}\cellcolor{blue}\bfseries{Priorité}\\
  \cr
  \hline
\end{tabular}\\


\section{Exigences de qualité}

\begin{tabular}{|>{\centering}p{1,5cm}|>{\centering}p{10cm}|>{\centering}p{3cm}|}
  \hline
  \color{white}\cellcolor{blue}\bfseries{Id}&
  \color{white}\cellcolor{blue}\bfseries{Intitulé}&
  \color{white}\cellcolor{blue}\bfseries{Priorité}\\
  \cr
  \hline
  EQ\_1&
  les messages d’erreurs
renvoyés par le compilateur doivent être explicites, les plus fines possibles et
surtout par rapport à ce qu'il s'est passé.&
  Important
  \cr
  \hline
\end{tabular}\\

\section{Exigences de réalisation}

\begin{tabular}{|>{\centering}p{1,5cm}|>{\centering}p{10cm}|>{\centering}p{3cm}|}
  \hline
  \color{white}\cellcolor{blue}\bfseries{Id}&
  \color{white}\cellcolor{blue}\bfseries{Intitulé}&
  \color{white}\cellcolor{blue}\bfseries{Priorité}\\
  \cr
  \hline
  ER\_1&
  La compilation produit un exécutable linux au format ELF&
  Indispensable
  \cr
  \hline
  ER\_2&
  Pas d'implémentation des cast&
  \cr
  \hline
  ER\_3&
  Pas d'implémentation de la générécité&
  \cr
  \hline
  ER\_4&
  Implémentation en code monolithique&
  Indispensable
  \cr
  \hline
  ER\_5&
  Implémentation en code partagé&
  Optionnel
  \cr
  \hline
  ER\_6&
  Pas d'attributs dans les interfaces. Seulement des méthodes.
  \cr
  \hline
\end{tabular}\\