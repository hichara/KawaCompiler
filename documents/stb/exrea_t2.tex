\begin{tabular}{|>{\centering}p{3cm}|>{\centering}p{7cm}|>{\centering}p{2.5cm}|>{\centering}p{3cm}|}
  \hline
  \color{white}\cellcolor{blue}\bfseries{Id}&
  \color{white}\cellcolor{blue}\bfseries{Intitulé}&
  \color{white}\cellcolor{blue}\bfseries{Acteur(s)}&
  \color{white}\cellcolor{blue}\bfseries{Priorité}\\
  \cr
  \hline
  EXR\_1&
  Nommage des fichiers de sortie&
  Utilisateur KAWAC&
  Secondaire
  \cr
  \hline
  EXR\_2&  
  Reconnaissance de la grammaire KAWA&
  KAWAC&
  Indispensable
  \cr
  \hline
  EXR\_3&
  Compilation d'application en monolithique&
  KAWAC&
  Indispensable
  \cr
  \hline
  EXR\_4&
  Gestion des mots clés&
  KAWAC&
  Indispensable    
  \cr
  \hline    
  EXR\_5&
  Fichier de sortie au format ELF&
  KAWAC&  
  Indispensable    
  \cr
  \hline
  EXR\_6&
  Reconnaissance et compilation de classes&  
  KAWAC&
  Indispensable    
  \cr
  \hline
  EXR\_7&
  Reconnaissance et compilation de classes abstraites&  
  KAWAC&
  Indispensable
  \cr
  \hline
  EXR\_8&
  Reconnaissance et compilation d'interfaces&  
  KAWAC&
  Indispensable    
  \cr
  \hline
  EXR\_9&
  Prise en charge du Polymorphisme&
  KAWAC&
  Indispensable    
  \cr
  \hline
  EXR\_10&
  Prise en charge de l'héritage de classe&
  KAWAC&
  Indispensable    
  \cr
  \hline
  EXR\_11&
  Prise en charge de l'implémentation d'interfaces&
  KAWAC&
  Indispensable    
  \cr
  EXR\_12&
  Mécanisme de constructeur&
  KAWAC \& Utilisateur KAWAC&
  Indispensable    
  \cr
  \hline
  EXR\_13&
  Mécanisme de finalisation&
  KAWAC \& Utilisateur KAWAC&
  Secondaire 
  \cr
  \hline
  EXR\_14& 
  Reconnaissance et définition de méthode&
  KAWAC \& Utilisateur KAWAC&  
  Indispensable    
  \cr
  \hline
  EXR\_15&
  Reconnaissance et définition d'attribut&
  KAWAC \& Utilisateur KAWAC&
  Indispensable    
  \cr
  \hline
  EXR\_16&
  Reconnaissance et définition des variables locales&  
  KAWAC \&  Utilisateur KAWAC&
  Indispensable    
  \cr
  \hline
  EXR\_17&
  Définition d'attributs statiques&
  Utilisateur KAWAC&
  Important
  \cr
  \hline
  EXR\_18&
  Définition d'attributs de constantes&
  Utilisateur KAWAC&
  Secondaire
  \cr
  \hline
  EXR\_19&
  Définition méthode \'a référence&
  Utilisateur KAWAC&
  Important
  \cr
  \hline
\end{tabular}\\
\newpage
\begin{tabular}{|>{\centering}p{3cm}|>{\centering}p{7cm}|>{\centering}p{2.5cm}|>{\centering}p{3cm}|}
  \hline
  EXR\_20&
  Définition méthode sans référence&
  Utilisateur KAWAC&
  Important
  \cr
  \hline
  EXR\_21&
  Définition méthode finale&
  Utilisateur KAWAC&
  Important
  \cr
  \hline
  EXR\_22&
  Définition méthode statique&
  Utilisateur KAWAC&
  Important
  \cr
  \hline

  EXR\_23&
  Héritage de méthode&
  KAWAC&
  Indispensable
  \cr
  \hline
  EXR\_24&
  Héritage d'attribut&
  KAWAC&
  Indispensable    
  \cr
  \hline
  EXR\_25&
  Redéfinition de méthode&
  Utilisateur KAWAC&
  Important
  \cr
  \hline
  EXR\_26&
  Surcharge des méthode&  
  Utilisateur KAWAC&
  Important
  \cr
  \hline
  EXR\_27&
  Mutabilité&
  KAWAC&
  Indispensable    
  \cr
  \hline
  EXR\_28&
  Concept de visibilité&
  Utilisateur KAWAC&
  Indispensable
  \cr
  \hline
  EXR\_29&
  Portée de variable&
  KAWAC&
  Important
  \cr
  \hline
  EXR\_30&
  Paramétrage de méthode&
  KAWAC&
  Important
  \cr
  \hline  
  EXR\_31&
  Mécanisme de contrôle&
  Utilisateur KAWAC& 
  Important
  \cr
  \hline
  EXR\_32&
  Mécanisme de bouclage&
  Utilisateur KAWAC&
  Important
  \cr
  \hline  
  EXR\_33&
  Bloc de classe&
  Utilisateur KAWAC&
  Indispensable
  \cr
  \hline
  EXR\_34&
  Bloc d'instructions&
  Utilisateur KAWAC&
  Indispensable
  \cr
  \hline
  EXR\_35&
  Bloc de boucle&
  Utilisateur KAWAC&
  Important
  \cr
  \hline
  
%%%%%%%%%%% A remplir %%%%%%%%%%%%%%
\end{tabular}\\
