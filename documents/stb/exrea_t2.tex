\begin{tabular}{|>{\centering}p{1,5cm}|>{\centering}p{13cm}|}%|>{\centering}p{3cm}|}
  \hline
  \color{white}\cellcolor{blue}\bfseries{Id}&
  \color{white}\cellcolor{blue}\bfseries{Intitulé}\\

  \cr
  \hline
  EXR\_1&
  On pourra choisir le nom du fichier à l'issue de la compilation.
  \cr
  \hline
  EXR\_2&
  KAWAC est capable de dire si le code est valide pour la grammaire de KAWA, et émettre des erreurs pour signaler la ligne et un moyen de résolution.
  \cr
  \hline
  EXR\_3&
  KAWAC pourra compiler des fichiers et fournir un exécutable qui n'a pas besoin de bibliothèques externes pour fonctionner. L'ensemble du code donné en entrée devra fournir une méthode main, qui sera le point d'entrée de l'application.
  \cr
  \hline
  EXR\_4&
  KAWAC pourra compiler des fichiers et fournir un exécutable. L'exécutable s'il le faut dépendra de ressources externes pour pouvoir fonctionner. Le code en entrée devra fournir une méthode main, qui sera le point d'entrée de l'application.
  \cr
  \hline
  EXR\_5&
  KAWAC pourra compiler des fichiers et fournir une biblithèque qui dépendra d'aucune bibliothèque externe. Les fichier passés en entrée doivent ne pas contenir de méthode main.
  \cr
  \hline
  EXR\_6&
  KAWAC pourra compiler des fichiers et fournir une nouvelle bibliothèque. La bibliothèque s'il le faut dépendra de ressources externes pour pouvoir fonctionner. Le code en entrée devra ne pas contenir de methode main.
  \cr
  \hline
  EXR\_7&
  KAWAC gère une table de mots clé contenant les mots réservés par la spécification de la grammaire de KAWA. 
  \cr
  \hline
  EXR\_8&
  Le compilateur utilisera le format ELF pour effectuer l'édition des liens.
  \cr
  \hline
  EXR\_9&
  Le programme exécutable intégrera un mécanisme permettant de gérer la mémoire au cours de l'exécution.
  \cr
  \hline
  EXR\_10&
  KAWAC est capable de reconnaître et compiler des classes écrites en KAWA.
  \cr
  \hline
  EXR\_11&
  KAWAC est capable de reconnaître et compiler des classes abstraites écrites en KAWA.
  \cr
  \hline
  EXR\_12&
  KAWAC est capable de reconnaître et compiler des interfaces écrites en KAWA.
  \cr
  \hline
  EXR\_13&
  KAWA permet d'utiliser des objets en utilsant des réferences de type différents, à la condition que le type statique soit un ancêtre du type dynamique. Seules les méthodes du type statique sont accéssibles.
  \cr
  \hline
  EXR\_14&
  KAWA autorise la dérivation de classe. KAWA ne gère pas l'héritage multiple de classe.
  \cr
  \hline
  EXR\_15&
  l'implémentation d'interface : KAWA autorise l'implémentation d'une ou plusieurs interfaces.
  \cr
  \hline
  EXR\_16&
  KAWA autorise la dérivation de classe. KAWA ne gère pas l'héritage multiple de classe.
  \cr
  \hline
  EXR\_17&
  KAWA autorise l'implémentation d'une ou plusieurs interfaces.
  \cr
  \hline


  EXR\_18&
  Chaque objet, pour être instancié, fournit une méthode qui permettra son intanciation. KAWAC fournira un constructeur si un n'est pas définit.
  \cr
  \hline
  EXR\_19&
  Chaque objet fournit une methode qui sera appelée lors de sa destruction par le garbage collector.
  \cr
  \hline
  EXR\_20&
  On peut déclarer un bloc d'instructions paramétrable s'exécutant s'il est appelé.
  \cr
  \hline
  EXR\_21&
  On peut allouer des espaces mémoires représentant les champs de chaque objet.
  \cr
  \hline
  EXR\_22&
  On peut définir des variables temporaires et de portées limitées. Les variables locales ne sont pas des attributs.
  \cr
  \hline
  EXR\_23&
  On peut définir des espaces mémoires associés aux classes. Les objets instanciant ou dérivant la classe où a été déclaré l'attribut et si la visibilité le permet, pointeront vers la même adresse pour cet attribut.     
  \cr
  \hline

\end{tabular}\\
\newpage
\begin{tabular}{|>{\centering}p{1,5cm}|>{\centering}p{13cm}|}

\hline
  EXR\_24&
  Un attribut constant ne peut être modifié après affectation. Il doit avoir été initialisé avant pour être utilisable par une autre opération que l'affectation.
  \cr
  \hline
  EXR\_25&
  En définissant un attribut avec le mot clé \'value\', on aura accès non pas a une réference vers un espace mémoire stockant les données, mais directement a un bloc de données. 
  \cr
  \hline
  EXR\_26&
  La méthode renvoie un bloc de données représentant le resultat.
  \cr
  \hline
  EXR\_27&
  La méthode renvoie une référence vers un espace mémoire contenant les données de l'objet renvoyé.
  \cr
  \hline
  EXR\_28&
  La méthode ne retourne rien.
  \cr
  \hline
  EXR\_29&
  La méthode ne peut être surchargée ou redéfinie.
  \cr
  \hline
  EXR\_30&
  La méthode peut être accéssible à partir du nom de la classe.
  \cr
  \hline
  EXR\_31&
  Une classe dérivant une autre copira toute les méthodes déja définit dans l'arborescence de ses ancêtres. Mais ne pourra y acceder que si la visibilité le permet.
  \cr
  \hline
  EXR\_32&
  Une classe dérivant une autre copira tout les attributs déja definit dans l'arborescenece de ses ancêtres si la visibilité le permet.
  \cr
  \hline
  EXR\_33&
  Une méthode peut être definie avec le nom d'une autre existence, en renvoyant le meme type de valeur, ainsi que les memes parametres.  
  \cr
  \hline
  EXR\_34&
  Une méthode peut être définit avec le nom d'une autre existence, en renvoyant le meme type de valeur, mais avec des paramètres différents.  
  \cr
  \hline
  EXR\_35&
  Les valeurs stockées dans les attributs et les variables pourront être réaffectées.
  \cr
  \hline
  EXR\_36&
  A l'aide des mots clés, on peut définir la visibilité des attributs et methodes des classes sur plusieurs niveaux. Privée, pour empêcher l'accès en dehors de la classe. Protégée, pour empêcher l'accès en dehors de la classe, sauf pour les héritiés de la classe ou de l'interface. Publique pour autoriser l'accès à tous.
  \cr
  \hline
  EXR\_37&
  Une variable n'est effective qu'à l'interieur du bloc à dans lequel elle a été déclarée.
  \cr
  \hline
  EXR\_38&
  Une méthode accepte des références ou des bloc de données en paramètre et les remplace aux bons endroits dans son bloc d'execution.  
  \cr
  \hline
  EXR\_39&
  KAWA admet des expressions conditionnelles.
  \cr
  \hline
  EXR\_40&
  KAWA admet des expressions de bouclage.
  \cr
  \hline
  EXR\_41&
  KAWAC gère une table de mots clé contenant les mots réservés par la spécification de la grammaire de KAWA.
  \cr
  \hline
  EXR\_42&
  La défintion du corps de l'objet se fera à l'intérieur d'un bloc délimité paramétré. Un bloc de classe ne peut en contenir un autre.
  \cr
  \hline
  EXR\_43&
  On peut définir une suite d'instructions délimité paramétré. Un bloc d'instruction peut contenir d'autres blocs.
  \cr
  \hline
  EXR\_44&
  On peut définir une suite d'instructions délimité paramétré. Les instructions contenues dans le bloc sont répétées tant qu'une certaines condtions sont remplies. Un bloc de bouclage peut contenir d'autres blocs.
  \cr
  \hline

\end{tabular}\\
